\documentclass{article}
\usepackage[utf8]{inputenc}
\usepackage{graphicx}
\usepackage{hyperref}
\usepackage{amsmath}
\usepackage{amssymb}
\title{Philo NB}
\author{Shinigami}
\date{2021}
\begin{document}
\maketitle
\textit{Note:}
\\\\\textit{Because this is a philosophy diary, I shall be recording what I believe to be true at the respective point in time. If I acknowledge that my opinion changes, it is my intention to change the contents of this notebook.}
\\\\\textit{Before starting, I will take the statement `Existence is' as true for granted. This is because denying it is the same as stating that 'red is not red', 'cold is not cold', or anything of that sort.}
\\\\\textit{Whenever I usethe word `infinity', it is with the philosophical and not the mathematical definition. For example, in the context of this book a ray cannot be called infinite but in a mathematical context it can.}
\newpage
\section*{Proofs}
\textit{\textbf{Prove that reality is infinite}}
\\
\\Here I shall prove by contradiction.
\\Let us asume that reality is finite.
\\Therefore reality has an end.
\\An end indicates the finishing of something's existance, that is, nonexistance after a point of time.
\\$\therefore$ Reality is infinite.
\\\\\textit{Note: I shalll be proving that time is false, and thus perhaps an illusion. Therefore, the concept of beginning and end are out of question themselves.}
\\
\\
\\
\textit{\textbf{Prove that time is unreal}}
\\There are many proofs to this. I shall record all that I am aware of at this time.
\\1
\\Time can be defined as an ordered set of events, with a cause, followed by an effect. In other words, it is an order of changes. Now a change has to have a cause by definition. Otherwise \textit{it} is constant. Now this cause will have to be another change, because otherwise whatever introduced by the change will be constant. Therefore, we cna keep on going, but it will never end. But one cause has to exist for the events. Therefore, it can be stated that time is an illusion
\\2
\\Time is not infinite, because it has a beginning. It may be endless (here, end is not used witht the mathematical definition but rather the understanding humans have in relation with time), though. But upon viewing it as a mathematical model, it has an endpoint (here I use the mathematical definition), and thus it is not infinite. And because reality is not finite, time is not reality. Thus time is an illusion.
\\\\LMFAO WAS I REALLY THAT STUPID TO CREATE A PHILOSOPHY NOTEBOOK? THE TRUTH IS CLEAR: IT IS EVERYWHERE, IT IS ULTIMATE, NOTHING IS NOWHERE, \textit{IT IS.}
\end{document}