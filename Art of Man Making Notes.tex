\documentclass{article}
\usepackage[utf8]{inputenc}
\usepackage{amsfonts}
\usepackage{amsmath}
\usepackage{amssymb}
\newcommand{\ch}[1]{\newpage\begin{center}\textit{\textbf{Talk #1}}\end{center}}
\newcommand{\tit}[1]{\textit{#1}}
\newcommand{\tbf}[1]{\textbf{#1}}
\newcommand{\p}{\\$\to$\hspace{2mm}}
\newcommand{\guru}{\tit{Guru}}
\newcommand{\bg}{\textit{Bhagavad Gita} }
\title{The Art of Man Making: Notes}
\author{ }
\begin{document}
\maketitle
\newpage \section*{Preface}
The War of Mahabharata is perhaps the greatest war India has seen since the birth of the civilization. Being a war between 2 superpowers (Hastinapura and Indraprastha), in effect this was \tit{the war}. The government of Hastinapura was terribly corrupt, run by criminals themselves, immoral to the extent that almost any method could be justified in order to take them out of power. Above this, by a treaty . For the sake the Sapta-Sindhu civilization, Indraprastha, after many attempts to establish peace diplomatically and without violence, declares war on Hastinapura. Before the war commenced, the leader of the army, Arjuna, gets devastated by the fact that he would have to kill his deeply loved ones during the war, in the battlefield. Krishna, the mentor and close friend of Arjuna, provides psychological treatment to Arjuna. He first lets Arjuna vent his problems, and then discusses the pragmatic side of Vedanta, i.e. the applications of Vedanta in society. Arjuna, because of his moral discipline, is able to have a clear mind when Krishna provides logical moral reasoning for the war and disproves by moral logic Arjuna's thoughts about running away, and preaches Sankhyan philosophy. Krishna then preaches the \tit{Bhakti Yoga}, literally the \tit{Path of Devotion}, which is the emotional, moral and personal discipline of the mind with surrender to an ideal, which by design calms the great warrior and further inspires him to do what he believes is morally correct. Now, having a clear mind, Arjuna is tempted to and proceeds with the discussion of the theory part of Vedanta itself, after which Krishna allows Arjuna to decide for himself what he needs to do, and Arjuna agrees to and gloriously fights the great war. The Gita, being a conversation between a patient and a doctor, serves as psychological treatment and guidance to action to those who seek it, and also a condensed insight into Vedanta philosophy. This document is a record of my interpretations of a book \tit{'The Art of Man Making'}, an insight on the Gita by saint-philosopher Swami Chinmayananda.
\newpage \section{Why Study The Gita?}
The Gita serves as an insight into the ethical side of Vedic philosophical study. Thus the Gita goes through the Vedic claims on how one must act upon their ideals \tit{(Karma Yoga)}, how one must contemplate and meditate upon their ideals \tit{(Dhyana Yoga}, also known as \tit{Hatha Yoga)}, how one must love and respect their idelals \tit{(Bhakti Yoga)}, and how one must think about and form their ideals \tit{(Gnyana Yoga)}.
\newpage \section{Urgency For The Study of The Gita}
This chapter serves its title fully, and can completely be summarised by it. Gurudev talks about the mental state of Arjuna, a brutal, intelligent, and righteous warrior, breaking down right before what is classified as a `Righteous War', which was to shape the destiny of India. At this time Krishna takes the decision to clear the veils in Arjuna's mind caused by his emotional state and allow him to make a choice with clarity of mind. Therefore Krishna lets Arjuna vent his emotions to him. Arjuna talks about how he does not want the joys of heaven that arise from killing his Gurus. After this, Krishna starts talking about how heaven does have a pennyworth of relevance, and all that matters is that Arjuna, as a warrior, fights for what is righteous and the best for his people. After gaining some amount of tranquility Arjuna starts to ask questions relating to deeper topics of Indian philosophy, and Krishna answers them whilst also providing an understanding of the theory and application of the Vedanta school of thought. At the end, Arjuna decides to fight the war.
\\
\\This chapter was written in the immediate post colonial era of India, although it still applies today and perhaps anyday.
\newpage \section{The Gita- Her Special Charm}
$\to$\hspace{2mm}Culture is important for the growth of society, and must find a good balance between total discipline and total free will.
\p Indian culture/thought was built off of the discussions and conclusions of all the philosophers collectively.
\p A common misinterpretation of Indian rituals (Hinduism) leads man to think that he must retire to the Himalayas leaving the world, and Arjuna also shared that opinion at a time, however Indian philosophy strictly emphasises the importance of working towards the society.
\p A phenomenon called the \tit{`suffering phobia'} is spoken of, where man is afraid of suffering. However, The Gita is the conversation between a man suffering from this phenomenon and another who had thoroughly studied Indian philosophy, and thus it can be a cure to this phobia.
\newpage \section{The Gita- A discovery of Life}
One cannot judge well from the perspective of an individual, as seen in many cases. The world is not a straightforward single sided coin. Arjuna was (very understandably) dreaded by the fact that he had to kill his own family. But morally, he did know that proceeding with war was correct. Being placed in a dilemma like this, perhaps in the worst situation for a moral man, Arjuna broke down. Krishna skillfully treated Arjuna by having a conversation with him which the world recognizes today as the Bhagavad Gita, or the Divine song. Therefore, it would be greatly helpful for the youth to read it, at a time when their ideals are strong, and at the same time have the energy to contribute to society.
\newpage \section{The Gita- A Scheme of Life}
The Gita serves as a record of the ancient Indian explorations of moral philosophy, internal and external, on topics like the morality of actions and individuals, and the progression of human society. Being a conversation about Vedanta exploring the ethical and practical aspects of it, the Gita covers the societal aspects of human actions, and proposes a solution to the problems faced by humans throughout life. The Gita states that the individual oneself is more important in the circumstances than the situations themselves, i.e. the solution to the problem lies in the man himself rather than outside. Therefore, the Gita would dictate that a man focuses on his mind rather than on the situations he faces to find peace. Using the analogy of functions, one can say that the type of result obtained is dependent on the value of the input. Vedanta would dictate that it is not the circumstances but man's relation with them that dictate his life, and thus, to live a good life one must enhance himself and his relation with the world around him. The ideal man according to the Ancient Indians was one who stood on his moral grounds without letting any circumstance affect him. Krishna, seeing the mental condition of Arjuna, attempts to make him achieve such a state through his conversation, turning successful.
\newpage \section{The Gita- The Art of Living And Striving}
The Gita serves as the record of a study of how man may live his life to the fullest, i.e. the subjective science. Therefore the study of the Gita is meant for everyone, and is ideally done with reference to daily life, and is advisable to be applied. The philosophers responsible for the contents of the Gita had the aim of improving one's own life and their findings have been recorded in the Gita.
\newpage \section{The Drama in Chapter One}
This chapter explains the contexts of the Gita. The Upanishads are the accounts of professional philosophical study and discussion, while the Gita is more of a layman's guide to the Vedanta school of thought. Krishna's target audience is Arjuna, a very skilled and heavily morally disciplined man who is at the verge of breakdown and is in complete confusion, dreaded by the idea of having to kill his kith and kin. Vyasa's target is India of his time, who believed that religion was for the people who leave society and go to the Himalayas. Therefore, by nature, the Gita serves as a guide to human society not only on Vedantic explanations of philosophical concepts but rather a pragmatic guidebook to acting in human society based on Vedanta, and is also more focussed on the youth because Arjuna's mental state was similar to that of teenagers and young adults. The target audience of this book is the post colonial youth, who had lost all their heritage due to the blunders Indians had been committing for the past century and had been brought up with dangerous superstitions projected as Indian culture. This had led to the country becoming not less than a hell, and is tragically not very different seventy years later.
\newpage \section{The Pride of Duryodhana}
Duryodhana, the leader of the Kaurava forces, had terribly underestimated the power of his opponents, because of the number of warriors on his side. However, man-power is only a part of a war. Indraprastha had the technology, integrity, and had much more intellectual capital which easily overpowered Hastinapura. Because of this, one of the greatest civil wars of the whole Indian civilization will end in eighteen days. Duryodhana, having been brought up with morals (despite having gone against them in every way he can) now feels weak in front of his enemies. He then censures his \guru \footnote[1]{I do not have a literal translation in english for the word. A \tit{Guru} is a good combination of a parent-figure, teacher and mentor.} by subtly indicating this\footnote[2]{Bhagavad Gita; Chapter 1, Verse 3}: \tit{Being a Bramhin\footnote[3]{In the profession system of ancient India, an individual was divided into one of four groups. Bramhins were the sect of people who pursue subjects of knowledge (philosophy, math, sciences, economics, etc.) as their profession. Duryodhana is a Kshatriya, a group of warriors. Drona, aside from his achievements in subjects of knowledge also had knowledge of warfare, enough to be elected as the \tit{Guru} of the princes of Hastinapura.}, despite your great knowledge of warfare, you are very timid. You cannot be expected to fight with valor. Do not worry, we have great warriors on our side.} Drona, his \tit{Guru}, being a man of great intellectual achievement, obviously understands this taunt, making it only very foolish of Duryodhana. This indicates the personality of the leader of the massive Hastinapuran army, arrogant and comparatively dim-witted. It is almost as if the Kauravas have lost before the battle started.
\newpage \section{The Crisis}
This chapter talks about the psychology of the leaders of both the armies, therefore I will divide my notes of this chapter into two sections.
\\\\\tit{Duryodhana: } The speech and mindset of the leader of the Hastinapura has already been explored in the previous chapter. Firstly, he passed a comment about the `cowardice’ of an important and respected warrior in the Kaurava army. He then asks his comrades to protect Bhishma, a vital member of the army, because if his intellect, war-experience, fighting skill, and also the reason a lot of the warriors chose to fight for Hastinapura rather than Indraprastha. Bhishma, the wise warrior, choses to bring back the integrity of the army by blowing the conch, distracting the Kaurava warriors whose zeal had been damaged by Duryodhana by blowing the conch, challenging Indraprastha to war. By the psychology of the leader and the loyalty and integrity of the people in the army, one can say that they lost before the battle was declared.
\\\\\tit{Arjuna: } Arjuna is facing his own problem, a psychological breakdown, having his ideals and personality built on morality but also dreaded by his need to kill his \tit{Guru} and Bhishma\footnote[1]{The genetic relation between Arjuna and Bhishma too complex and irrelevant to be bothered with here, so I will not explain it. Bhishma was the one who brought up the Pandavas and Kauravas, serving as a strong parent figure, teacher, and mentor to Arjuna (Almost like a Guru).} and others of his kith and kin. He asks his charioteer, his best friend and mentor, Krishna, to take his chariot to the middle of the battlefield, so he is able to look at the Kauravas. Krishna then says: \tit{Behold, O Partha\footnote[2]{Another name of Arjuna.}! The Kurus gathered together.} Krishna, making a statement for the first time since the start of Sanjaya's narration\footnote[3]{The point of view taken here is of Sanjaya, who is aiding te blind Dhritarashtra, the blind father of the Kauravas and the puppet king of Hastinapura by speaking of the details of the war. The choice of taking this point of view is quite wise of Vyasa, given that he would be the most unbiased person in the war. (Some amount of bias towards the side of the Pandavas is evident in his vocabulary as he intends to pursue the king biased by his biological ties with his children and blindly (quite literally and metaphorically) accepting the decisions of his children into ordering them to call truce. However, Vyasa holds a similar bias, trying to inspire the masses into adapting the virtues of the sons of Pandu. It can also be argued that out of all the people present, Sanjaya, by circumstance and role in the war is the most unbiased character.)}, known today as the Gita, tries to get Arjuna involved in actively making a judgement. What goes on in Arjuna's mind is described very well in this book. What normally happens, is the stimulus reaches the brain, man realises it, and based on his personality (a compilation of action thoughts, experiences), and the thoughts he is presented with (by his unconscious brain), he makes a decision. Arjuna, completely confused, is left without an answer.
\newpage \section{The Psychological Break Up}
Arjuna has requested to be placed in the centre of the battlefield, to view both his army and his opponent's army clearly. Krishna, while directing the chariot of the great archer towards the centre, draws Arjuna's attention to the warriors by saying: \tit{Behold, O Partha! The Kurus gathered together.} The sight Arjuna sees is, to him at that time, incredibly tragic. Seeing Bhishma, his grandsire; Drona, his \guru; his cousins; and all other kinsmen; he breaks down. Being rivals, morally wanting to stop almost every action the Kauravas perform, he always wanted to have a fair battle against the Kauyravas, whose cunning found a way to cheat every time. However, they dare not do so against this army\footnote[1]{With Indraprastha having the intel and skill of warriors like Arjuna, Krishna, Yudhishtira, Dhrishtadyumna; and some good technology, it is very hard to argue that the Hastinapuran forces which were lead by the uncouth Duryodhana had much of an advantage.}; and therefore Arjuna finally got his chance. However, only a few moments before commencement, he realises the impact of such a duel; that he would have to bear the blood of his own kinsmen on his hands. He shows symptoms of anxiety state neurosis, and openly describes them to Krishna. The traumatised warrior then pays heed to his escapism for sometime, a quality nurtured by natural selection. Arjuna, at the moment, wants to escape this situation of high tension. His values and principles, for the time being, are numb, and his intellect, hugely malfunctions. Arjuna experiences all symptoms of breakdown. Arjuna even indicates that he sees bad omens against the Kauravas. This shows how clouded he is, given that he is a man who can appreciate the study of philosophy and possessing a very powerful intellect, being a legendary strategist, and that his skill with the bow is talked about even today, centuries later. At this point, Krishna accurately treats Arjuna. He does not stop forceful overflow of Arjuna's emotions; tearing through the dam of his intellect. He does not even utter a word, letting all of Arjuna's feelings and sentiments flow through. The intellectual does not open his mouth until the mighty archer is ready to listen; ready to discuss with logic. Until Arjuna has a mind clear enough to be advised on morals. In a situation where Arjuna could take unbiased decisions on his own. A little later, the same brocken and unstable man will fight with his life on the line for the sake of his ideals, and emerge victorials.
\newpage \section{The Arjuna Disease}
Arjuna now suffers of a personality break up; something met by characters across many stories, fiction/non-fiction, when they come to realise something which opposes something  they have held very close to themselves. Arjuna, who has built himself on the basis of moral ideals, has lived his whole life under his moral disciplines. Now, however, he is pit against the people he loves the most by the same ideals. At this situation, he cannot take it. Swami Chinmayananda calls it the \tit{Arjuna Disease}, the situation of personality breakdown. The relevence of this to the crisis of the war is this: Arjuna serves as the leader to the Indraprasthan forces. He is the leader of the army (if I am not mistaken, even if so he is a very important leader to the forces). Therefore his agitation is in effect the breakdown of the leader and therefore the instbility of a force of people who are fighting to restore justice in their country. Given the leadership of the sons of Dhristarashtra, the impact of this on the country is far worse. When Arjuna is done, the king of Dwaraka starts to speak. Starting with the second chapter of the Gita (the first is an introduction with the narration of the situation and Arjuna venting out) which he uses as a prologue to talk about the moral ideas he will cover in later chapters, he speaks about how Arjuna should only work towards his ideals, and stay with his emotions. He talks about the problems of anger and other emotions and how they lead to the destruction of a man. The \tit{Arjuna Disease} is the shattering of personality. This can occur to an individual or a group of people. This can be seen in many situations, in war, in crime investigation, in family, and even within one's mind. The \bg serves as a treatment to such a situation, because it is the conversation in between a depressed and shattered warrior and a philosopher-psychologist trying to uphold morality.
\newpage \section{Why Should We Act?}
`Action is the insignia of life.' Here I am quoting Swami Chinmayananda, from `Kindle Life'. Therefore, action is inevitable. Therefore, what uthe focus is is over the quality of action. An action acheives a certain quality based on the ideals it is motivated by. Therefore here it is adviced to act with moral and pure intentions, without filth.
\newpage \section{Look At The Problem}
Before man acts, the brain is supposed to observe, understand and evaluate the given circumstance. As seen previously; we need to organise the internal system (with the lack of a more precise term) in order to attain harmony. Here, the ability of the intellect to observe, analyse and evaluate is conditioned by the \tit{vasanas}\footnote[1]{Literally means scent; used to indicate the tendencies and personality that make up the core of a human individual built with experiences.}. Therefore, \tit{vasanas} promoting morality will give raise to a moral action; \tit{vasanas} of curisity will promote investigation and reasoning, etc. Therefore, in order for man to purify his actions, he needs to purify his \tit{vasanas}. This is the art of \tit{Karma-Yoga}.
\end{document}