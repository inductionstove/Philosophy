\documentclass{article}
\usepackage{amsmath}
\usepackage{amssymb}
\newcommand{\tit}[1]{\textit{#1}}
\title{Account of Philosophical Works}
\begin{document}
\maketitle
\newpage
\section{Disproof of Space-Time}
Before we proceed to start any arguement, let us first define space and time. \tit{Space can be called as the container; something that contains.} This definition also has a problem as we do not have a formal definition of the word `contain'. \tit{To contain,} I define for the sake of this arguement, \tit{is the ability to assert the existence of something else while it itself exists.}I must also clarify that I am not dealing with physical space specifically; I am not studying space-time as a physicist would. I am talking about the broader concept of space, not necessarily drawing from thhe observable world. Mathematical concepts of three dimentional space and the linear time; and the approach of physics towards this concept are forms of the currently given definition of space. Let is go deeper. The space of physics, a derivation of the mathematical model of dimentional space, has the ability to contain, has the ability to assert the existence of our world. our world exists only because of this space, and cannot otherwise. The mathematical model of space is made up of infinitely small point(s) (a 0 dimentional space would have a singularity, only one point would exist) called co-ordinates. We can say that each point of 3d mathematical space is mapped to a particular infinitely small unit of existence in physical space via a one-to-one function. Coming back to the point, we see that space is the ability to assert the existence of another entity. Therefore, duality can exist in space. In fact, duality can exist \tit{only} in space. Before making this claim, we will have to define duality. \tit{Duality is the existence of more than one entity.} An entity is different from another either by \tit{implying the nonexisene of that entity} (for example, green is not red because the existence of green implies nonexistence of red) , for example, morality and immorality; or by being a composite of other units that make it up, where one or more entity making up one of the composite entity has a direct contradiction in another. For example, a brick painter red is not the same as a brick painted green because dispite the fact that they have similar properties, one entity that makes the green brick is the absense of `red-ness' and the eistence `green-ness', so to speak, and the entity that makes up the red brick is `red-ness' and the absense of `green-ness'. Therefore, both of the entities are different, or rather, contradictions. The primary principle of logic states that two contradicting entities/statements cannot host the property of being true simultaniously. However, space allows this. One can say that this much is enought to prove that space does not exist, but I am not so sure of that therefore I will go further. When we defined space, we said that space could be called as the entity that implies the existence of duality. Now, this is the same thing as saying that the existence of space=the existence of dualiy (this is because the definition space itself was an entity implies the existence of something else); therefore space=duality (even if only one thing exists in space, the space and the entity are two different things.) This means that neither can the existence/nonexistence of space in itself prove the existence/nonexistence of duality, nor can the existence/nonexistence of duality in itself prove the existence/nonexistence of space. Therefore, an external factor is needed to prove the existence/nonexistence of it. I shall bring up the primary (and in fact only) principle of logic here, again, that two contradicting entities cannot be true simultaniously. This disproves the existence of duality. Unlike a normal circumstance, the existence of duality \tit{cannot} be justified by space or vice verca, because we have shown that both are exactly the same thing. Therefore, I conclude, that if existence exists, it is a singularity. 
\end{document}